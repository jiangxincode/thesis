\begin{preface}

复杂网络的研究可上溯到20世纪60年代对ER网络的研究。90年后代随着Internet
的发展,以及对人类社会、通信网络、生物网络、社交网络等各领域研究的深入,
发现了小世界网络和无尺度现象等普适现象与方法。对复杂网络的定性定量的科
学理解和分析,已成为如今网络时代科学研究的一个重点课题。

在此背景下,由于云计算时代的到来,本文针对面向云计算的数据中心网络基础
设施设计中的若干问题,进行了几方面的研究。本文的创造性研究成果主要如下
几方面:

\begin{enumerate}
\item 基于簇划分的思想,提出并设计了WarpNet网络模型。该网络模型基于随机
  散列,以节点微路由链接多种散列分布,实现网络互联。并对网络的带宽等指
  标进行理论分析并给出定量描述。最后对比了理论分析、仿真测试结果,并在
  实际物理环境中进系真实部署,通过6节点的小规模实验以及1000节点虚拟机的
  大规模实验,表明该模型的理论分析、仿真测试与实际实验吻合,并在网络性
  能、容错能力、伸缩性灵活性方面得到较大提升。
\item 提出DS小世界模型并构造SIDN网络,解决了把小世界理论应用于数据中心
  网络布局构建中的最大度限制问题。分析了在带有最大度限制约束下,所构成
  网络的平均直径、网络总带宽、对故障的容错能力等各项网络参数。理论分析
  与仿真实验证明,SIDN网络具有很好的扩展能力,网络总带宽与网络规模成近
  似线性增长的关系;具有很强的容错能力,链路损坏与节点损坏几乎无法破坏
  网络的联通性,故障率对网络性能的影响与破坏节点/链路占总资源比率线性相
  关。
\item 分析了无尺度网络在数据中心网络构建应用中的理论方面问题。在引入节
  点最大度限制之后,给出无尺度网络的各项网络参数。并进一步分析了交换机
  节点以及计算节点两种角色在不同比率的组合下对网络性能的影响,给出最高
  性价比的比率参数。最后通过理论分析与仿真实验证明,在引入了无尺度现象
  之后,提高了网络的聚类系数,从而显著的提升了网络的性能。

\item 针对网络模型研究这一类工作的共性,设计构造通用验证平台系统。以海
  量虚拟机和虚拟分布式交换机的形式,实现了基于少量物理节点,对大规模节
  点的模拟。其模拟运行的过程与真实运行在实现层面完全一致,运行的结果与
  真实环境线性相关。除为本文所涉若干网络模型提供验证外,可进一步推广到
  更为广泛的领域,为各种网络模型及路由算法的研究工作,提供分析、指导与
  验证。
\end{enumerate}

复杂网络的研究可上溯到20世纪60年代对ER网络的研究。90年后代随着Internet
的发展,以及对人类社会、通信网络、生物网络、社交网络等各领域研究的深入,
发现了小世界网络和无尺度现象等普适现象与方法。对复杂网络的定性定量的科
学理解和分析,已成为如今网络时代科学研究的一个重点课题。

在此背景下,由于云计算时代的到来,本文针对面向云计算的数据中心网络基础
设施设计中的若干问题,进行了几方面的研究。本文的创造性研究成果主要如下
几方面:

\begin{enumerate}
\item 基于簇划分的思想,提出并设计了WarpNet网络模型。该网络模型基于随机
  散列,以节点微路由链接多种散列分布,实现网络互联。并对网络的带宽等指
  标进行理论分析并给出定量描述。最后对比了理论分析、仿真测试结果,并在
  实际物理环境中进系真实部署,通过6节点的小规模实验以及1000节点虚拟机的
  大规模实验,表明该模型的理论分析、仿真测试与实际实验吻合,并在网络性
  能、容错能力、伸缩性灵活性方面得到较大提升。
\item 提出DS小世界模型并构造SIDN网络,解决了把小世界理论应用于数据中心
  网络布局构建中的最大度限制问题。分析了在带有最大度限制约束下,所构成
  网络的平均直径、网络总带宽、对故障的容错能力等各项网络参数。理论分析
  与仿真实验证明,SIDN网络具有很好的扩展能力,网络总带宽与网络规模成近
  似线性增长的关系;具有很强的容错能力,链路损坏与节点损坏几乎无法破坏
  网络的联通性,故障率对网络性能的影响与破坏节点/链路占总资源比率线性相
  关。
\item 分析了无尺度网络在数据中心网络构建应用中的理论方面问题。在引入节
  点最大度限制之后,给出无尺度网络的各项网络参数。并进一步分析了交换机
  节点以及计算节点两种角色在不同比率的组合下对网络性能的影响,给出最高
  性价比的比率参数。最后通过理论分析与仿真实验证明,在引入了无尺度现象
  之后,提高了网络的聚类系数,从而显著的提升了网络的性能。

\item 针对网络模型研究这一类工作的共性,设计构造通用验证平台系统。以海
  量虚拟机和虚拟分布式交换机的形式,实现了基于少量物理节点,对大规模节
  点的模拟。其模拟运行的过程与真实运行在实现层面完全一致,运行的结果与
  真实环境线性相关。除为本文所涉若干网络模型提供验证外,可进一步推广到
  更为广泛的领域,为各种网络模型及路由算法的研究工作,提供分析、指导与
  验证。
\end{enumerate}

\vspace{1cm}
\begin{flushright}
韦小宝\\
2013年夏于南京大学
\end{flushright}

\end{preface}